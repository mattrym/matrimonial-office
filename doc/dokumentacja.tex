\documentclass[10pt,a4paper]{article}
\usepackage[T1]{fontenc}
\usepackage[utf8]{inputenc}
\usepackage{graphicx}
\usepackage[polish]{babel}
\usepackage{hyperref}
\usepackage[paper=a4paper,margin=1in]{geometry}
\usepackage{tabularx} 
\usepackage{helvet}
\usepackage{amsfonts}
\usepackage{amsthm}
\usepackage{mathtools}
\usepackage{algpseudocode}
\usepackage{tikz}
\usepackage{tkz-graph}
\usepackage{listings,xcolor}
\renewcommand{\familydefault}{\sfdefault}
\setlength{\parindent}{0pt}
\newtheorem{theorem}{Twierdzenie}
\newtheorem{lemma}{Lemat}
\newtheorem{conculsion}{Wniosek}

\begin{document}
\begin{titlepage}
\newgeometry{top=1in,bottom=1in,right=1.5in,left=1.5in}
\begin{center}
{\fontsize{14}{12}\selectfont Wydział Matematyki i Nauk Informacyjnych Politechniki Warszawskiej}

\end{center}

\vspace{1cm}
\begin{center}
\includegraphics[width=0.3\textwidth]{images/logo.png}
\end{center}
\vspace{3cm}

\begin{center}
\textbf{{\fontsize{26}{12}\selectfont Systemy ekspertowe}}

\vspace{2cm}
\textbf{{\fontsize{22}{12}\selectfont Dokumentacja projektowa}}
\vspace{1cm}

\textbf{{\fontsize{13.5}{12}\selectfont Aleksandra~Bułka, Aleksandra~Byczyńska Jakub~Czyżniejewski, Mateusz~Rymuszka}}

\vspace{6cm}
\textbf{{\fontsize{13.5}{12}\selectfont \today}}
\end{center}  
\end{titlepage}

{\fontsize{13.5}{12}\selectfont
\tableofcontents
\vspace{1cm}
{\renewcommand{\arraystretch}{2.0}

\newpage

\section{Wprowadzenie i opis problemu}

% TODO: to w sumie przekopiować z prezentacji w dużym stopniu, plus to tego może jakieś przykłady?

\section{Szczegóły techniczne}

% TODO: dwa zdania o niczym

\subsection{Technologia}

% TODO: coś tam o prologu napisać i tej implementacji SWI

\subsection{Struktura}
	
% TODO: opisać strukturę, w jaki sposób dane są uniezależnione od interfejsów, itp. jakiś schemacik może?
	
\subsection{Szczegóły implementacyjne}

% TODO: kilka podrozdziałów opisujących co się dzieje, pewnie co nieco o logice klasycznej, rozmytościach, w jaki sposób jest to wszystko liczone i potem prezentowane

\section{Instrukcja użytkowania}

% TODO: tutaj parę screenów, opisać co do czego, przykłady użycia itp.
	
\end{document}