\documentclass[12pt]{article}
\usepackage{polski}
\usepackage[utf8]{inputenc}
\usepackage{graphicx}
\usepackage{amsmath}
\usepackage{float}
\usepackage{multirow}
\usepackage{longtable}
\usepackage{makecell}
\usepackage{hyperref}
\usepackage[margin=1in]{geometry}
\usepackage{algorithmic}
\usepackage[most]{tcolorbox}
\definecolor{code}{RGB}{210, 210, 210}
\geometry{margin=1in}

\usepackage[yyyymmdd]{datetime}
\renewcommand{\dateseparator}{--}

\newcommand{\currentVersion}{1.0}

\usepackage{titlesec}

\setcounter{secnumdepth}{4}

\titleformat{\paragraph}
{\normalfont\normalsize\bfseries}{\theparagraph}{1em}{}
\titlespacing*{\paragraph}
{0pt}{3.25ex plus 1ex minus .2ex}{1.5ex plus .2ex}

\begin{document}
\begin{titlepage}
		\centering
		\includegraphics[scale=0.15]{logo_mini.png}
		
		{\scshape\LARGE Politechnika Warszawska \par}
		{\scshape\Large Wydział Matematyki i Nauk Informacyjnych\par}
		\vspace{1.5cm}
		{\huge\bfseries \itshape Systemy Ekspertowe Biuro~matrymonialne\par}
		\vspace{0.5cm}
		{\Large \bfseries \itshape Raport do projektu\par}
		\vspace{2cm}
		{\Large\itshape Aleksandra Bułka\par}
		\vspace{0.15cm}
		{\Large\itshape Aleksandra Byczyńska\par}
		\vspace{0.15cm}
		{\Large\itshape Jakub Czyżniejewski\par}
		\vspace{0.15cm}
		{\Large\itshape Mateusz Rymuszka\par}
		\vfill
		wersja \currentVersion
		\vfill
		\today
		\vfill
	\end{titlepage}
	\section{Cel projektu}
	Celem projektu było zaprojektowanie i zaimplementowanie w języku Prolog systemu ekspertowego usprawniającego parowanie osób przez biuro matrymonialne. Parowanie miało odbywać się w oparciu o posiadaną wiedzę na temat cech i atrybutów klientów obecnych w bazie danych oraz w oparciu o odpowiedzi klienta prowadzącego interakcję z programem na pytania dotyczące jego preferencji. Zbieżność odpowiedzi klienta oraz cech osoby z bazy danych miały prowadzić do sparowania. Program został podzielony na 3 logiczne części: cześć użytkownika, w której klient wypowiada się na temat swoich preferencji, część administracyjna, w której zarządzający bazą danych użytkowników może edytować dane nt. klientów oraz część wnioskująca, w której zawarto reguły parowania użytkowników.
	\section{Moduł użytkownika}
Moduł użytkownika jest uruchamiany poprzez wywołanie kolejno \textbf{swipl ./src/user\_ui.pl} oraz \textbf{main.}.
Interakcja z programem polega na odpowiadaniu na szereg pytań, których przykłady zamieszczono poniżej:
	\begin{verbatim}
	
'What should be his/her age?'
'What should be his/her gender?'
'What should be his/her height?'
'What location (lon, lat) should he/she be living in'
'What should be his/her attractiveness level?'
'What should be his/her income (yearly)?'
'What level of education should he/she have?'
'What occupation should he/she have?'
'What should his/her religious affiliation be?'
'What should be his/her sexual orientation'
'How important (from 1 (not important) to 5 (very important) are the following 
life priorities to you: family, work, hobbies, travelling,
wellness, spirituality, partying, self-development'
	\end{verbatim}

Po przeprowadzeniu wywiadu z klientem, program wyświetla listę najlepiej dopasowanych osób z bazy danych wraz z podstawowymi informacjami na ich temat.
	
Niektóre z zadawanych pytań pozwalają jedynie na odpowiedzi TAK/NIE, jednak jest część pytań, np. pytanie o wiek, na które udzielona może być odpowiedź z pewnego zakresu wartości. Wprowadzenie tego typu pytań powoduje, że konieczne staje się pewne \textit{rozmycie} relacji dopasowania. Ostateczny wynik obliczania dopasowania dwóch osób to nie jest zatem \textbf{0} lub \textbf{1}, lecz raczej liczba rzeczywista z przedziału o tych krańcach.
\newpage
\section{Moduł wnioskowania}
Odpowiedzialny za wyznaczenie stopnia dopasowania dwóch osób jest moduł wnioskowania. Jak już wspomniano, w ramach systemu funkcjonują 2 rodzaje pytań - zamknięte i \textit{rozmyte}. W przypadku pytań zamkniętych logika systemu jest prosta - dla osób, które nie są zgodne na poziomie pytań zamkniętych, dopasowanie wynosi 0, wpp. 1. Dla pytań \textit{rozmytych} natomiast, zdefiniowano w systemie dedykowane dla każdego atrybutu funkcje zbieżności wartości atrybutu $f: A \times A \rightarrow [0, 1]$. Każda z tych funkcji ma własność $f(x, x)=1$. Intuicyjnie - funkcja mówi jak bardzo dwie wartości tego argumentu są ze sobą zgodne i oczywiście dwie identyczne wartości muszą być ze sobą w pełni zgodne. Dopasowanie dwóch osób jest wyliczane jako iloczyn dopasowań poszczególnych par wartości atrybutów.
\section{Moduł zarządzania}
Moduł zarządzania bazą osób jest dostępny poprzez uruchomienie kolejno \textbf{swipl ./src/expert\_ui.pl} oraz \textbf{main.}
Administrator ma dostęp z poziomu menu do szeregu trybów zarządzania bazą danych. Menu wygląda następująco:
\begin{verbatim}
### Matrimonial office - expert interface
 Choose an action:
  0. Quit program
  1. Print all people from database
  2. Print information about a single person
  3. Save changes to database
  4. Add a person to database
  5. Edit a person from database
  6. Remove a person from database
  7. Remove all the people from database
  8. Change feature of a person
  9. Remove feature of a person
\end{verbatim}
Nazwy poszczególnych zakładek dobrze opisują ich funkcje. Przejść pomiędzy zakładkami dokonuje się poprzez podanie odpowiedniego numeru zakładki z klawiatury. Administrator ma możliwość edycji danych dowolnej osoby z bazy oraz jej usunięcia. Możliwe jest także dodawanie nowych osób oraz wyświetlanie pełnej listy osób lub szczegółów każdej osoby z osobna.
\end{document}